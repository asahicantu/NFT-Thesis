%!TEX root = ../../main.tex
\section{Existing Approaches/Baselines}
All the components of the system aforementioned form the baseline for this thesis work, however little has been done in permissioned Blockchain systems that allow data management as a \ac{NFT} digital asset with \ac{IPFS}. There is however more research on its usage for public Blockchain networks and other industry purposes as the mentioned below. 


\subsubsection{e-Health}
Healthcare systems maintain electronic medical records using the centralized storage models, potentially compromising user privacy. Potential threats include unauthorized access to critical information such as identity details, diseases from which a patient suffers, and misuse of patients' data. \ac{IPFS} and blockchain technology, a distributed off-chain storage of medical data, can be created while preserving patient privacy.

\begin{itemize}
    \item \cite{9027313} proposes a framework to facilitate easy access to medical data by authorized entities while preserving consistency, integrity, and availability.
    
    \item \cite{CHEN2021102771} suggests a system able to handle \ac{EHR}s. for diabetes disease detection that provides an earlier detection of this disease by using various machine learning classification algorithms and securing the information. Blockchain, and \ac{IPFS} are used to collect patient's health information via wearable sensor devices.
    
    \item \cite{kumar2021decentralized} proposes in the same manner a decentralized system for medical data storage that allows the community to securely share information and remove it from central systems since the high cost of data breaches, cyber attacks and the implicit cost of restructuring the \ac{IT} infrastructure would prevent the progress in medical research and creation of new medical alternatives. 
    
    \item VAHAK\cite{9162738} is an interesting project aiming to provide a low-latency, secure and reliable infrastructure via Ethereum smart contracts, 5-G and \ac{IPFS} to allow communication and navigation of \ac{UAV}s and improve the air-distribution of medical supplies in areas hard to reach or in crisis.
    
    \item HealChain\cite{8865388} is a decentralized \ac{DMS} for Mobile Healthcare Using Consortium Blockchain, that aims to build a system where security prevails when patients share medical data wirelessly.
    
    \item \cite{9066212} is another system implementation to handle \ac{EHR} criptographically and decentralized.
    
    \item \cite{9759580} Has implemented a case study in Nepal for an Ethereum and \ac{IPFS} based Application model to record and share patient healdh information.
    
\end{itemize}

\subsubsection{Patents and intellectual property}
\begin{itemize}
    \item \cite{bamakan2022patents} intends to apply \ac{NFT}-based patent framework in public Blockchain networks and divers \ac{DFS}  to the intellectual property to promote transparency and liquidity for innovators willing to commercialize their inventions or be funded.
    
    \item \cite{agyekum2019digital} propose a digital media copyright and content protection using \ac{IPFS} and Blockchain to expedite traditional digital copyright solidification and rights which are time consuming and labor-intensive using Hyperledger Fabric and the management of digital fingerprints for patents, ensuring immutability and provenance.
    
    \item \cite{8591029} models a data Integrity and confidentiality framework using encryption, smart contracts and Apache Isis for the automatic audit trial. 
 
\end{itemize}

\subsubsection{Automotive}
\cite{nizamuddin2021blockchain} proposes an \ac{IPFS}/Ethereum blockchain-based system for the auto insurance sector to solve the problem of fraud and latency and bureaucracy that customers face whenever acquiring insurance for for vehicles. Insurance prevent the industry from progressing due to the liabilities and difficulties it presents in many situations to understand it.

\subsubsection{Identity and Authorization}
\cite{9240966} present a solution for Blockchain-based \ac{MPA} for Accessing IPFS Encrypted Data using Ethereum smart contracts and proxy re-encryption algorithms. It incorporates reputation mechanisms in the smart contracts to rate the oracles based on their malicious and non-malicious behaviors. Thus, this system can solve insider's attack problem by ensuring that a single authority or party is not acting alone. 

\subsubsection{Tourism}
\cite{demirel2021smart} have created a model with Blockchain and \ac{IPFS} integration for post pandemic economy for the tourism industry to fill the gap in the existing methods related to the use of the \ac{IoT}, devices with smart contracts without a need for intermediaries for the reservations and services secured by Blockchain. The authors have created a booking system with a unique smart contract between customers and hotels, including all services that a customer may need and elimination commission fees as well as reception costs.


\subsubsection{Agriculture}
\cite{8718621} have created a Blockchain-Based Soybean Traceability in Agricultural Supply Chain by  utilizing smart contracts to govern and control all interactions and transactions among all the participants involved within the supply chain ecosystem. All transactions are recorded and stored in the ledger with links to \ac{IPFS} and thus providing to all a high level of transparency and traceability into the supply chain ecosystem in a secure, trusted, reliable, and efficient manner.