%!TEX root = ../thesis.tex

\chapter{Conclusions}
\label{ch:conclusion}

\section{Final comments}
In line with the original hypothesis stated in \ref{analysis_hyp} it is possible to confirm that the construction of decentralized systems is possible and will create yet unforeseen possibilities to manage information and provide:
\begin{itemize}
    \item System scalability
    \item Modularity
    \item Self Governance by smart contract agreement
    \item Privacy and security
    \item Mutual cooperation
\end{itemize}

Systems such as this can be governed or not by a central authority. Depending on the business needs and industrial purposes different scenarios could be developed and simulated.

Key findings through the research and development of this thesis project are:
\subsection{Blockchain and DLT}.
Data integrity and consistency safeguarding. Decentralized system nowadays allow very easily to verify data origin, integrity and non-repudation principle  establishes new ways of working and contributing for the development and research of technology.

\subsection{Security}
The created infrastructure has several layers of security and implicit elements to protect parties and their data from improperly accessing information. 
\subsubsection{\ac{CA}s}
Different parties can choose to rely over a central or multiple \ac{CA}s. As long as there is a common agreement previously established by smart contracts and common consensus, it will be possible to create highly resistant and resilient systems to malicious attacks. The modularity of Blockchain allows the cooperation of parties in benefit of the whole system, therefore rejecting undesirable or suspicious behaviour.

\subsubsection{Private channels}
Private channels in Hyperledger Fabric allow organizations to create a subsystem inside the framework, where sub smart contracts can be created to allow privacy between a single, two or more entities in different regions of the world.
Therefore, while every organization can hold a copy of the ledger and data, only those with the right privileges can be allowed to interact with the hidden rules.

\ac{IPFS} offers several layers of security, data can be encrypted and directed by smart contract instructions and even by the security layers set into the built system. 

\subsection{Governance}
Organizations can self in their best interest to preserve business functionality and choose the most suitable way to manage data. That being said, they will be able to choose which \ac{CA} is the best, what rules in the smart contract can be implemented and under which conditions entities can mint or transfer \ac{NFT}s.


\section{Future Work}
It is intended for anyone willing to extend and expand the functionality of this thesis work to move within the following points:
\subsection{System integration}
This thesis project can complement the work made by \cite{akbarAli}, which explains how shared data can be used to perform workflows for different purposes such as Big data analytics and Machine learning. One key advantage of this implementation relies on the fact that participants will have no direct access to shared data itself, but to a framework where it can be processed and manipulated to generate different models and insights. When connecting both systems, it will be possible to create and encapsulate information, self data encryption can grant that even different organizations own a copy of the blockchain and \ac{IPFS} network, they cannot read the information unless they do it directly from the proposed application.

\subsection{NFT and Smart Contract extension}
An extension of \ac{ERC}-721 smart contract implemented in \ref{ERC271ChainCode} can be easily extended to create new business rules such as:

\subsubsection{\ac{NFT} delegated ownership and transfer}
Not only one user, but multiple entities could possess a digital asset, sharing a percentage of such element. Therefore new rules can be suggested to interact, protect and manipulate information.

\subsubsection{Consensus}
New consensus mechanisms can be generated to incentivize the usage of the network. This project contains for example rules to rank information and create a reputation level for the organizations, which can increase the value of the minted \ac{NFT}. In other scenarios can be possible to create digital ownership by data origin, geographical location and mechanisms of burning so it cannot be used by other parties.
 
\subsubsection{Economics through Tokenization}
A very interesting approach to explore is the creation of economic tokens integrated with the \ac{NFT} system. Such tokens and in the same form that \ac{ETH} does with the Ethereum platform, every piece of data can be linked to another unity of tokens where once transferred its value in tokens can be transferred as well. The token-value of data can increase as its ranking of "valuable information" increases, or organization reputation does.
Depending on those economic mechanisms, different companies could be able to generate royalties and incentivize the usage of the system. Furthermore.

\subsubsection{Multi-system integration}
Furthermore, this project can be integrated with other public Blockchain platforms, and allow the issuance or reading of  Ethereum smart contracts, enabling its execution in the internal network. The possibilities are endless and adaptable to specific business needs.