%!TEX root = ../../main.tex
\section{Outline}
The thesis projects consists of the following chapters:
\subsection{Chapter 1. Introduction}
\begin{itemize}
  \item A brief explanation of decentralized systems and how they emerged with the invention of Bitcoin.
Through the creation of smart contracts, the issuance of fungible and non-fungible tokens will be possible, both in a proto-version and in a more advanced form with the evolution of decentralized systems.
Overview of NFT-based systems, their original applications, and their potential for industrial use
\item Further explanation of permissioned blockchain systems and the role of Hyperledger Fabric as an open-source platform for industrial applications.
\item A brief presentation of IPFS as a decentralized file system and its potential to use decentralized content-based storage solutions in private networks. These two systems: "Hyperledger Fabric" and "IPFS," are merged to create an infrastructure able to mint NFT representing digital data assets with different capabilities able to solve modern industry data sharing and solve trust, authenticity, and security problems.
\end{itemize}
\subsection{Chapter 2. Related Work}
\begin{itemize}
    \item Demonstration how how permissioned blockchain systems developed and deployed using "Hyperledger Fabric" have helped industries expedite processes and open collaboration with other parties.
    \item Related work and applications of IPFS as a decentralized file system for the Web 3.0. Brief explanation of how this has helped to empower end users and how its application for the industry can create independence and full ownership of data with similar reliability as the provided by cloud companies.
    \item NFT systems based in the Ethereum and other blockchain networks.
    \item Similar approaches through the implementation of these systems in the public blockchain
\end{itemize}
\subsection{Chapter 3. Approach}
This chapter explains how "Hyperledger Fabric" and "IPFS" System were used together to incorporate a permissioned blockchain system to share data in a secure and self trusted manner. 
\begin{itemize}
    \item A demonstration of a custom smart contract developed specifically to hold the necessary instruction for NFT issuance.
     \item The concept of "proof of ownership" and "proof of authenticity" are incorporated as consensus mechanisms to avoid data counterfeiting and incentivize its curation, usability and reliability through multiple evaluation mechanisms.
\end{itemize}
\subsection{Chapter 4. Experimental Evaluation}
With all the elements provided in \ref{ch:approach:main}, a whole infrastructure is assembled and its functionality is demonstrated through the implementation of Linux containers as virtual servers emulation the whole systems. 
back-end and front-tend systems have been build in addition as this process allows to show the system in its full potential and propose further system escalation as required.
\subsection{Chapter 5. Discussions}
From proposed solution in \ref{ch:eval} future work and other consensus approaches are discussed. Its scalability and evolution can be set in place for different industry applications beyond big data sharing.
\subsection{Chapter 6. Conclusions}
This chapter explains how after the development and incorporation of two unrelated systems it is possible to create reliable infrastructures with the potential to change the way companies currently use, store and share information inside and among other organizations.
