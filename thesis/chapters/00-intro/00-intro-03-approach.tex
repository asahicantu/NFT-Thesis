%!TEX root = ../../main.tex
\section{Approach and Contributions}
%Give a brief summary of your overall approach. Summarize the specific contributions that you made in this thesis (implementation, empirical results, analysis, etc.).

This thesis work uses Hyperledger Fabric permissioned Blockchain framework as a base platform to implement an ERC-721 smart contract extension for the creation and certification of digital assets, which for the aim of this thesis is represented in data. 
It also implements an distributed data lake infrastructure through the usage of a private IPFS network, which allows storage of information in a reliable and trustless manner. When this data is linked to the ERC-721 smart contract, an NFT asset is created, and single entity properties like ownership, authenticity, transfer, royalties, etc, can be exposed to all its participants.

It demonstrate through the creation of a back-end and front-end example applications the potential of its usage through a simulated token minting and data sharing environment among different known parties where the unique source of trust is a central authority and the previous agreement of the rules (consensus protocol) through which the data can be generated, shared and transferred in the same manner as if a crypto asset was transferred. Since the environment has been implemented using Docker Containers Technology, its scalability and security are granted and can be easily implemented.  The results can be visualized in  the following github repository, the program can be downloaded and deployed in any computer system with specific requirements
\begin{itemize}
    \item ERC-721 protocol creation and extension
    \item Consensus mechanism
    \item Infrastructure and Blockchain environment in Hyperledger Fabric
    \item REST API in the back-end for common collaboration with the blockchain
    \item Front new Web application as a simulation of  work and data generation through the blockchain
    \item Creation of a private IPFS network and data persistence schema for common data sharing
\end{itemize}

A detailed analysis of the systems infrastructure and collaboration, the interpretation of roles and mechanisms that an organization will play to grant plain participation and collaboration over the blockchain systems in Hyperledger
Data persistence and availability in the IPFS environment.

Generation of a proof of authenticity and non repudiation of data, consensus protocol, trough the exploration of the blockchain and closes application system.