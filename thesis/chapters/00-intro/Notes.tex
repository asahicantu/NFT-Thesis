The Bitcoin standard includes a proto-scripting language for implementing a self-trusting system, in which all participants agree on a set of rules:
\begin{itemize}
    \item A maximum of 21,000,000 bitcoins can ever exist.
    \item In the first 210,000 blocks, each block generated 50 bitcoins (about every 10 minutes).
Following the first 210,000 blocks, the number of bitcoins generated will halve to 25 (the number keeps splitting by 2, approximately every four years). The system will generate the last Bitcoin around year 2140.
    \item A deflationary economy is the purpose of this new economic and democratic virtual agreement. Over time, the value of the currency is likely to increase (corresponding to scarce resources).
 
\end{itemize}





Bitcoin's proto-scripting language generated new ways where economic tokenization \cite{malinova2018tokenomics}, finance and token usability can help implement for the first time trustless systems of worldwide cooperation. In consequence, many projects are emerging from the original idea of Blockchain, created to remediate and innovate the problem of trust, counterfeiting and centralization. New standards for code as a "proof of trust" created the possibility to develop and compile source code in the blockchain (better known as smart contracts). The first system of its kind in fully extending a Blockchain to host compiled code was the Ethereum Project \footnote{Ethereum was conceived in 2013 by programmer Vitalik Buterin and implemented on its first version in 2015\cite{Ethereum30:online} able to execute instructions which everyone can read, acknowledge and be certain that no matter the circumstances or the parties involved, it will work according to the specified conditions

 however, little has been researched so far for its application in the industry, where supply chain and 

  and the basis for different projects 
Industries can nowadays benefit from blockchain systems to record and track data in more trustful ways and allow cooperation. However, the problem of privacy, ownership, and security emerges when trying to implement such systems. In this thesis project, a schema and system design is proposed to record unique ownership of certain assets in the blockchain (datasets ownership) and provide the infrastructure to allow its usage or deny it depending on the agreements of the system through the issuance of NFT's for specific datasets.





Bitcoin's proto-scripting language generated new ways where economic tokenization \cite{malinova2018tokenomics}, finance and token usability can help implement for the first time trustless systems of worldwide cooperation. In consequence, many projects are emerging from the original idea of Blockchain, created to remediate and innovate the problem of trust, counterfeiting and centralization. New standards for code as a "proof of trust" created the possibility to develop and compile source code in the blockchain (better known as smart contracts). The first system of its kind in fully extending a Blockchain to host compiled code was the Ethereum Project \footnote{

 however, little has been researched so far for its application in the industry, where supply chain and 

  and the basis for different projects 
Industries can nowadays benefit from blockchain systems to record and track data in more trustful ways and allow cooperation. However, the problem of privacy, ownership, and security emerges when trying to implement such systems. In this thesis project, a schema and system design is proposed to record unique ownership of certain assets in the blockchain (datasets ownership) and provide the infrastructure to allow its usage or deny it depending on the agreements of the system through the issuance of NFT's for specific datasets.



Many projects are emerging from the original idea of Blockchain, specially created to remediate the economic boundaries that banks are still unable to solve. New consensus standards in source code to represent a "proof of trust" created the possibility to develop and compile immutable agreements potentially improbable to break when all the parties agree on its contents and liabilities. The term "smart contract" accrued by Nick Szabo in 1997 \cite{szabo1997formalizing}, came to life with the first computer peers joint the Bitcoin Network.

Through this protocols and smart contracts the concepts of digital fungible asset and  non-fungible asset permitted a new way to transact and send value by altering the Blockchain ledger.

Bitcoins are in its essence fungible tokens\footnote{An object is fungible when and if it is identical to others and thus can be replaced without any loss (mutual interchangeability).}, but it also allows since its creation the minting of "Coloured coins, better known as non-fungible tokens (NFT).\footnote{As bitcoins scripting language allows the coding of basic instructions, it is possible to represent asset manipulation through the exchange of assets. This makes possible to create the concept of unicity in a ledger, making a transaction or an asset irrepeatable.}





% \subsection{Chapter 2. Token economics}
% \subsection{Chapter 3. Fungibility and non fungibilityy}
% \subsection{Chapter 4. Fungibility and non fungibilityy}
% \subsection{Chapter 5. IPFS as a decentralized File system}
% \subsection{Chapter 6. Hyperledger fabric as a permissioned blockchain network}
% \subsection{Chapter 7. Implementation of the Digital Asset System }
% \subsection{Chapter 8. The consensus protocol }
% \subsection{Chapter 9. Proof of ownership }
% \subsection{Chapter 10. Simulation and program }
% \subsection{Chapter 11. Future work }

A data breach occurs when sensitive, confidential, or protected data is exposed to an unauthorized environment. It is possible to be affected by it due to a hacker attack, an inside job by an employee who is currently or previously employed by the organization, or as a result of unintentional data theft or loss.

Data breaches can involve information leakage, also known as exfiltration—unauthorized copying or transmission of data without affecting sources. In other cases, breaches incur a complete loss of data, as in ransomware attacks, which involve hackers encrypting data to deny access to the data owner.

In other words, hackers or employees release or leak sensitive data in a data breach. As a result, perpetrators might lose or use the data for various malicious purposes.





Blockchain-based distributed database systems are emerging technologies that aim to provide users with the necessary tools to perform tasks and computations without the assistance of an intermediate entity, to build self-trust systems where rules are agreed upon beforehand, and to create the basis for an ecosystem that can be used by everyone. Moreover, users may interact democratically and be confident that committed transactions will be immutable, truthful, and irrefutable.
Bitcoin is the first and most famous example of a worldwide distributed blockchain system. 

At the time of this thesis' publication, bitcoin had demonstrated robustness, reliability, and evidence that new ways of trust/collaboration were possible over the last 14 years (since its implementation in 2009)\cite{nakamoto2008bitcoin} to alleviate the economic boundaries that banks have yet to resolve. This is very important because since its creation numerous Blockchain-based technologies have been created in regards to provide solutions to financial and industrial problems. 


% In this thesis project, a schema and system design is proposed to record unique ownership of certain assets in the blockchain (datasets ownership) and provide the infrastructure to allow its usage or deny it depending on the agreements of the system through the issuance of NFT's for specific datasets.

% This thesis work used a permissioned Blockchain approach in Hyperledger Fabric create an infrastructure for the management of Digital Assets in a decentralized ledger owned by multiple parties and managed by a \ac{NFT} smart contract and using a \ac{DFS} with \ac{IPFS} as a storage system.